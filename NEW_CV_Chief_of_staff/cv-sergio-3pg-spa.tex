%% Curriculum Vitae Profesional - Sergio Muñoz de Alba Medrano, MVZ
%% Versión en Español - Enero 2026 - 3 Páginas Optimizado
%% Compilar con: pdflatex cv-sergio-3pg-spa.tex

\documentclass[10pt,letterpaper,sans]{moderncv}

% moderncv style
\moderncvstyle{banking}
\moderncvcolor{blue}

% Encoding and fonts
\usepackage[utf8]{inputenc}
\usepackage[T1]{fontenc}
\usepackage{lmodern}

% Margin adjustment - márgenes más ajustados para 3 páginas
\usepackage[scale=0.80,footskip=35pt]{geometry}
\setlength{\hintscolumnwidth}{2.2cm}

% Reduce name and title font size
\renewcommand*{\namefont}{\fontsize{18}{20}\mdseries\upshape}
\renewcommand*{\titlefont}{\fontsize{12}{14}\mdseries\upshape}

% Reduce spacing
\setlength{\parskip}{2pt}

% Personal data
\name{Sergio}{Muñoz de Alba Medrano}
\title{Médico Veterinario Zootecnista}
\address{Mérida, Yucatán}{México}{}
\email{smunozam@gmail.com}
\phone[mobile]{+52~999~200~5550}
\homepage{tinyurl.com/linkedinSMAM}
\extrainfo{Dirección de Operaciones y Programas | Chief of Staff}

%----------------------------------------------------------------------------------------
\begin{document}

\makecvtitle

%----------------------------------------------------------------------------------------
% RESUMEN PROFESIONAL
%----------------------------------------------------------------------------------------

\section{Resumen Profesional}

\cvitem{}{Directivo en operaciones con \textbf{30+ años de experiencia} en programas públicos de gran escala, equipos multidisciplinarios (320+ personas) y presupuestos de \textbf{250+ millones de MXN}. Se desempeña como \textbf{socio tipo Chief of Staff} para la alta dirección, alineando estrategia con operación. Expertise en ejecución de programas, desarrollo organizacional, diseño de procesos y relación con actores clave en sectores público, no lucrativo y privado.}

%----------------------------------------------------------------------------------------
% COMPETENCIAS CLAVE
%----------------------------------------------------------------------------------------

\section{Competencias Clave}

\cvitem{Operaciones y Liderazgo}{Equipos multidisciplinarios (320+ personas), 22 oficinas regionales, objetivos claros y seguimiento de resultados}
\cvitem{Gestión de Programas}{Portafolios multi-programa: presupuestos 250+ millones MXN, 45,000+ beneficiarios, alineación estratégica}
\cvitem{Gestión Financiera}{Supervisión presupuestaria, asignación de recursos, apoyo en auditorías, transparencia en entornos públicos y no lucrativos}
\cvitem{Desarrollo Organizacional}{Consejos, comunidades de práctica, equipos interfuncionales para colaboración y gobernanza}
\cvitem{Procesos y Calidad}{Diseño de SOPs, participación ISO 9000, optimización de flujos y mejora continua}
\cvitem{Relaciones Estratégicas}{Gobiernos federal/estatal, agencias internacionales, productores, OSC, sector privado}
\cvitem{Sistemas Digitales}{Plataformas web, tableros, sistemas de seguimiento de datos (GOB.mx v3, acuerdos bilaterales)}

%----------------------------------------------------------------------------------------
% EXPERIENCIA PROFESIONAL
%----------------------------------------------------------------------------------------

\section{Experiencia Profesional}

\cventry{05/2025--Presente}{Consultor Independiente}{SADER Yucatán}{Mérida}{}{Fortalecimiento de sistemas operativos, gestión de datos y gobernanza en sanidad animal. Lideró diseño e implementación del \textbf{Centro de Consulta de Acuerdos Zoosanitarios} (sistema de seguimiento bilateral). Coordinó actores federales, estatales e internacionales.}

\cventry{08/2020--04/2025}{Gerente de Proyectos}{Youcatan Land}{Mérida}{}{Proyectos estratégicos en conservación y desarrollo rural. Obtuvo concesión estatal para \textbf{Tabi, Santuario de Selva Maya} (1,350 ha). Diseñó Programa de Estrategias para transición agroecológica con actores locales.}

\cventry{09/2017--12/2025}{Profesor}{Universidad Marista de Mérida}{Mérida}{}{Docencia en programa de Gestión de Recursos Naturales. Impartió sistemas de producción y ganadería sostenible. Asesoró proyectos aplicados vinculando conceptos técnicos con retos operacionales.}

\cventry{10/2013--08/2017}{Economista Agrícola}{The Nature Conservancy}{Mérida}{M-REDD+}{Implementó sistemas silvopastoriles en 8 ranchos piloto. Secretario Ejecutivo de Comunidad de Práctica de Ganadería Sustentable. Coordinó equipos técnicos, productores e instituciones socias.}

\cventry{08/2012--02/2013}{Encargado del Despacho}{SAGARPA}{Yucatán}{}{Máxima autoridad agrícola federal en el estado. Lideró 320+ servidores públicos en 22 oficinas. Coordinó todos los programas agrícolas federales en Yucatán. Relaciones institucionales con Gobierno del Estado.}

\cventry{10/2005--09/2013}{Subdelegado de Planeación}{SAGARPA}{Yucatán}{}{Gestionó programa Procampo (120M MXN a 45,000 productores). Supervisó 250+ millones MXN anuales en programas federación-estado. Implementó mecanismos de cumplimiento y transparencia. Planeación estratégica estatal.}

\cventry{08/2002--09/2005}{Jefe de Distrito}{SAGARPA}{Yucatán}{Ticul}{Operaciones regionales de programas federales. Coordinó apoyos para 18,000+ productores anualmente. Administró 148 servidores públicos. Estableció 18 Consejos Municipales.}

\cventry{08/1998--07/2001}{Propietario}{AGROPHARMA}{Yucatán}{}{Microempresa farmacéutica veterinaria. Incrementó ventas anuales de 150K a 2M MXN. Lanzó 15 nuevos productos. Gestión de proveedores, inventarios, controles de calidad.}

\cventry{01/1994--12/1997}{Gerente de Nuevos Productos y Exportación}{LAPISA}{Michoacán}{}{Diseñó departamento de exportación desde cero. Operaciones en 13 países (Latinoamérica, Asia). Elaboró expedientes regulatorios y registros zoosanitarios. Participación en ISO 9000.}

%----------------------------------------------------------------------------------------
% FORMACIÓN Y HABILIDADES TÉCNICAS
%----------------------------------------------------------------------------------------

\section{Formación Académica}

\cventry{2004--2007}{Maestría en Gobierno y Políticas Públicas (cursos)}{UADY}{Mérida}{}{}
\cventry{1994--1997}{Maestría en Administración de Empresas (cursos)}{Universidad del Valle de Atemajac}{La Piedad, Michoacán}{}{}
\cventry{1984--1988}{Licenciatura en Medicina Veterinaria y Zootecnia}{UNAM}{Ciudad de México}{}{}

\section{Idiomas}

\cvitemwithcomment{Español}{Nativo}{}
\cvitemwithcomment{Inglés}{Bilingüe}{Profesional}
\cvitemwithcomment{Francés}{Intermedio}{}

\section{Habilidades Técnicas}

\cvitem{Desarrollo Web y Apps}{HTML, CSS, JavaScript, Vue.js, Firebase, estándares GOB.mx v3; aplicaciones para herramientas de calidad y cálculos financieros}
\cvitem{Herramientas de Desarrollo}{Visual Studio Code, Git/GitHub, Node.js}
\cvitem{Datos y Analítica}{Tableros de control, herramientas estadísticas, QGIS, integración de datos para reportes}
\cvitem{Productividad}{MS Office, Google Workspace, gestión colaborativa de documentos}
\cvitem{Herramientas IA}{GitHub Copilot y asistentes IA para redacción de documentos, análisis y reportes}

\end{document}