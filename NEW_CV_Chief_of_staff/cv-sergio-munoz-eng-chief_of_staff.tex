%% Professional CV - Sergio Muñoz de Alba Medrano, DVM
%% English Version - December 2025
%% Compile with: pdflatex cv-sergio-munoz-eng.tex

\documentclass[10pt,letterpaper,sans]{moderncv}

% moderncv style
\moderncvstyle{banking}
\moderncvcolor{blue}

% Encoding and fonts
\usepackage[utf8]{inputenc}
\usepackage[T1]{fontenc}
\usepackage{lmodern}

% Margin adjustment
\usepackage[scale=0.85,footskip=40pt]{geometry}
\setlength{\hintscolumnwidth}{2.5cm}

% Reduce name and title font size
\renewcommand*{\namefont}{\fontsize{20}{22}\mdseries\upshape}
\renewcommand*{\titlefont}{\fontsize{14}{16}\mdseries\upshape}

% Personal data
\name{Sergio}{Muñoz de Alba Medrano}
\title{~\\Doctor of Veterinary Medicine}
\address{Mérida, Yucatán}{Mexico}{}
\email{smunozam@gmail.com}
\phone[mobile]{+52~999~200~5550}
\homepage{tinyurl.com/linkedinSMAM}
\extrainfo{Operations \& Program Management | Public Administration | Organizational Development}

%----------------------------------------------------------------------------------------
\begin{document}
\makecvtitle

%----------------------------------------------------------------------------------------
%	PROFESSIONAL SUMMARY
%----------------------------------------------------------------------------------------
\section{Professional Summary}

\cvitem{}{Operations and program management leader with \textbf{30+ years of experience} running large-scale public programs, complex initiatives, and multidisciplinary teams across government, non-profit, and private organizations. Proven track record overseeing \textbf{budgets of 250+ million MXN}, leading teams of \textbf{300+ staff} and coordinating 20+ regional offices, while implementing structures, processes, and systems that improve execution, transparency, and compliance. Acts as a \textbf{Chief of Staff–style partner} to senior leadership, aligning strategy with day-to-day operations, organizing information flows, and ensuring that plans, people, and resources move in the same direction. Brings a mission-driven mindset with deep experience in public administration and sustainable development, and is ready to apply these management capabilities in diverse sectors beyond conservation.}

%----------------------------------------------------------------------------------------
%	CORE MANAGEMENT COMPETENCIES
%----------------------------------------------------------------------------------------
\section{Core Management Competencies}

\cvitem{Operations Leadership}{Direction of multidisciplinary teams of up to \textbf{320+ staff} across 22 regional offices, with clear objectives, monitoring, and follow-up.}
\cvitem{Program Management}{Execution of multi-program portfolios with \textbf{250+ million MXN} annual budgets and \textbf{45,000+ beneficiaries}, ensuring alignment with strategic goals.}
\cvitem{Financial Stewardship}{Oversight of program budgets, resource allocation, documentation, and support for audits and transparency in public and non-profit environments.}
\cvitem{Organizational Development}{Creation of councils, communities of practice, and cross-functional teams to improve collaboration, governance, and knowledge sharing.}
\cvitem{Process \& Quality}{Design and implementation of SOPs, participation in ISO 9000 systems, and continuous improvement of workflows and controls.}
\cvitem{Stakeholder Management}{Liaison with federal and state governments, international agencies, producers, NGOs, and private sector partners, using diplomacy to align interests.}
\cvitem{Digital \& Data Systems}{Use of web platforms, tracking systems, and dashboards (GOB.mx v3, traceability and agreement-tracking systems) to manage information and support decision-making.}

%----------------------------------------------------------------------------------------
%	PROFESSIONAL EXPERIENCE
%----------------------------------------------------------------------------------------
\section{Professional Experience}

\cventry{05/2025--Present}{Independent Consultant (External Advisor)}{SADER Yucatán}{Mérida}{}{
Collaborating with the State Representative's team to strengthen operational systems, data management, and governance for animal health and bilateral agreements.
\begin{itemize}
\item Led the design and implementation of the \textbf{"Centro de Consulta de Acuerdos Zoosanitarios"} (ceso-aphis-yuc.web.app), a bilingual tracking system for CESO and APHIS-USDA/SENASICA agreements and compliance.
\item Structured data models, standardized information fields, and defined workflows to ensure accurate registration, tracking, and retrieval of bilateral agreements and supporting evidence.
\item Coordinated federal, state, and international stakeholders to align procedures, documentation requirements, and update cycles, improving traceability and audit readiness.
\item Developed dashboards and reporting views to monitor agreement status, pending actions, and historical records, supporting decision-making and institutional transparency.
\end{itemize}}

\cventry{08/2020--04/2025}{Project Manager}{Youcatan Land SAPI de CV}{Mérida}{}{
Managed strategic projects linking conservation, land stewardship, and rural development, with strong emphasis on coordination with authorities and communities.
\begin{itemize}
\item Led the administrative and technical process to secure the state concession for \textbf{"Tabi, Santuario de Selva Maya"} (1,350 ha), coordinating documentation, requirements, and follow-up with government agencies.
\item Designed the \textbf{"Integrated Strategies Program for Agricultural Sustainability"} for Blue Core A.C., defining objectives, indicators, and monitoring mechanisms for agroecological transition in rural communities.
\item Engaged with local stakeholders to align project activities with community needs, conservation goals, and regulatory frameworks.
\end{itemize}}

\cventry{09/2017--12/2025}{Professor, Natural Resources Management Programme}{Universidad Marista de Mérida}{Mérida}{}{
University faculty member in the Natural Resources Management program, contributing to the training of future professionals in sustainable production and resource management.
\begin{itemize}
\item Taught "Analysis and Evaluation of Animal Production Systems" and "Design of Sustainable Livestock Systems" to senior undergraduate students.
\item Designed course syllabi, evaluation criteria, and project-based learning activities integrating productivity, sustainability, and regulatory compliance.
\item Advised students on applied projects, helping them connect technical concepts with real-world operational and organizational challenges.
\end{itemize}}

\cventry{10/2013--08/2017}{Agricultural Economist, M-REDD+ Strategy}{The Nature Conservancy}{Mérida}{Alianza México REDD+}{
Supported the implementation of low-carbon rural development strategies, coordinating technical teams, producers, and partner institutions in the field.
\begin{itemize}
\item Coordinated the implementation of silvopastoral livestock systems in 8 pilot ranches, organizing technical assistance, work plans, and monitoring activities with producers.
\item Served as Executive Secretary of the Sustainable Cattle Production Community of Practice, managing information flows, meetings, and documentation among multiple organizations.
\item Contributed to reporting and knowledge management for M-REDD+ activities, ensuring that field information and lessons learned were captured and organized for program use.
\end{itemize}}

\cventry{08/2012--02/2013}{Acting Head of Delegation}{SAGARPA}{Yucatán}{}{
\textbf{Highest federal agricultural authority in the State of Yucatán}, temporarily acting as Regional Director and overall operations lead for the institution in the state.
\begin{itemize}
\item Provided strategic and operational leadership to a team of \textbf{320+ public servants} distributed across \textbf{22 regional offices}, ensuring continuity of services and program delivery.
\item Coordinated all federal agricultural program operations in Yucatán, aligning multiple initiatives, teams, and resources under a single operational direction.
\item Led institutional relations with the State Government, producer organizations, and sector stakeholders, acting as main liaison and representative of the federal institution in the state.
\item Oversaw implementation of national policies at state level, monitoring performance, addressing operational issues, and ensuring adherence to federal guidelines and standards.
\end{itemize}}

\cventry{10/2005--09/2013}{Subdelegate for Planning and Rural Development}{SAGARPA}{Yucatán}{}{
Acted as operations and planning lead for federal agricultural programs in Yucatán, translating national policies into executable plans at state level and coordinating multi-program portfolios with significant budget and impact.
\begin{itemize}
\item Managed the \textbf{Procampo program}, overseeing the annual flow of \textbf{120 million MXN} to more than \textbf{45,000 producers}, including beneficiary database management, eligibility validation, and payment execution.
\item Oversaw the execution of \textbf{250+ million MXN annually} in concurrent federal–state programs under the \textbf{"Alianza para el Campo"} framework, aligning budgets, project pipelines, and reporting requirements with annually issued operating rules, framework and specific agreements, and their technical annexes documenting physical and financial targets.
\item Implemented procedures and control mechanisms to ensure legal compliance, fiscal discipline, and transparency in the use of public resources.
\item Led strategic planning processes for rural development in the State, integrating data from multiple programs to support decision-making, monitoring, and evaluation.
\item Coordinated with state authorities, producer organizations, and other stakeholders to resolve operational bottlenecks, keep programs aligned with on-the-ground needs, and oversee the systematic monthly integration of statistical data (production, planted and harvested areas, etc.) from local offices (CADER and Rural Development Districts) in collaboration with the Agricultural and Fisheries Information Service (SIAP).
\end{itemize}}

\cventry{08/2002--09/2005}{Head of Rural Development District, Ticul}{SAGARPA}{Yucatán}{}{
Regional operations lead for federal rural development programs in the Ticul district.
\begin{itemize}
\item Coordinated subsidy programs for over \textbf{18,000 producers annually}, ensuring proper registration, eligibility verification, and follow-up of support delivered.
\item \textbf{Managed a team of 148 public servants}, organizing workflows, distributing tasks, and monitoring performance across multiple municipalities.
\item Established \textbf{18 Municipal Councils for Sustainable Rural Development}, creating formal structures for participation, information exchange, and program oversight.
\end{itemize}}

\cventry{08/1998--07/2001}{Owner}{AGROPHARMA}{Yucatán}{Veterinary Pharmaceutical Microenterprise}{
Entrepreneur and general manager of a veterinary pharmaceutical microenterprise, responsible for overall operations, commercial strategy, and financial performance.
\begin{itemize}
\item Increased annual sales from \textbf{150,000 to 2 million MXN}, expanding the client base and product penetration in the regional market.
\item Launched \textbf{15 new products}, coordinating formulation, registration, packaging, pricing, and go-to-market strategy.
\item Managed supplier relationships, inventory, and basic quality controls to ensure product availability and reliability for veterinary clients.
\item Oversaw administrative, financial, and commercial processes, including billing, collections, and customer service.
\end{itemize}}

\cventry{01/1994--12/1997}{New Product Development \& Export Manager}{LAPISA, SA de CV}{Michoacán}{Veterinary Pharmaceutical Laboratory}{
Held combined responsibilities in new product development and international export operations for a veterinary pharmaceutical manufacturer.
\begin{itemize}
\item \textbf{Designed and established the export department} from scratch, including procedures, documentation templates, and coordination mechanisms with internal areas and international clients.
\item Initiated commercial operations in \textbf{13 countries} across Latin America and Asia, managing logistics, shipment coordination, and regulatory requirements for product entry.
\item Led the preparation of regulatory dossiers and served as liaison with agriculture ministries and regulatory agencies in multiple countries to obtain zoosanitary registrations.
\item \textbf{Participated in ISO 9000 certification processes}, developing SOPs for international product registration and export logistics, and aligning operations with quality management standards.
\item Coordinated cross-functional teams (R\&D, production, quality, sales) to ensure that new products and export operations met both technical and regulatory criteria.
\end{itemize}}

%----------------------------------------------------------------------------------------
%	EDUCATION
%----------------------------------------------------------------------------------------
\section{Education}

\cventry{2004--2007}{Master's in Government and Public Policy (coursework completed)}{Universidad Autónoma de Yucatán}{Mérida}{}{}
\cventry{1994--1997}{Master's in Business Administration (coursework completed)}{Universidad del Valle de Atemajac}{La Piedad, Michoacán}{}{}
\cventry{1984--1988}{Bachelor's in Veterinary Medicine and Animal Science}{Universidad Nacional Autónoma de México}{Mexico City}{Professional License}{}

%----------------------------------------------------------------------------------------
%	LANGUAGES
%----------------------------------------------------------------------------------------
\section{Languages}

\cvitemwithcomment{Spanish}{Native}{}
\cvitemwithcomment{English}{Bilingual Proficiency}{Professional oral and written communication}
\cvitemwithcomment{French}{Intermediate}{}

%----------------------------------------------------------------------------------------
%	TECHNICAL SKILLS
%----------------------------------------------------------------------------------------
\section{Technical Skills}

\cvitem{Web \& App Development}{HTML, CSS, JavaScript, Vue.js, Firebase, GOB.mx v3 standards; development of small web applications for quality tools, financial calculations, and pension planning.}
\cvitem{Development Workflow}{Visual Studio Code, Git \& GitHub (version control, repositories for web apps and utilities), basic use of Node.js tooling.}
\cvitem{Data \& Analytics}{Data analysis, dashboards, basic use of statistical and geospatial tools (QGIS), integration of program data for reporting and decision-making.}
\cvitem{Productivity Tools}{Microsoft Office, Google Workspace; collaborative document management and presentations for senior stakeholders.}
\cvitem{AI \& Automation}{Use of Artificial Intelligence tools (including GitHub Copilot and other assistants) to draft documents, structure information, and accelerate analysis and reporting tasks.}

\end{document}
