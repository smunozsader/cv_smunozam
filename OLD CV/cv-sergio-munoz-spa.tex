%% CV Profesional - MVZ Sergio Muñoz de Alba Medrano
%% Versión Español - Diciembre 2025
%% Compilar con: pdflatex cv-sergio-munoz-spa.tex

\documentclass[10pt,letterpaper,sans]{moderncv}

% Estilo moderncv
\moderncvstyle{banking}
\moderncvcolor{blue}

% Codificación y fuentes
\usepackage[utf8]{inputenc}
\usepackage[T1]{fontenc}
\usepackage{lmodern}

% Ajuste de márgenes
\usepackage[scale=0.85,footskip=40pt]{geometry}
\setlength{\hintscolumnwidth}{2.5cm}

% Reducir tamaño del nombre y título para que quepan en sus líneas
\renewcommand*{\namefont}{\fontsize{20}{22}\mdseries\upshape}
\renewcommand*{\titlefont}{\fontsize{14}{16}\mdseries\upshape}

% Datos personales
\name{Sergio}{Muñoz de Alba Medrano}
\title{~\\Médico Veterinario Zootecnista}
\address{Mérida, Yucatán}{México}{}
\email{smunozam@gmail.com}
\phone[mobile]{+52 999 200 5550}
\homepage{tinyurl.com/linkedinSMAM}
\extrainfo{Especialista en Administración Pública, Políticas Agropecuarias y Desarrollo Rural}

%----------------------------------------------------------------------------------------
\begin{document}
\makecvtitle

%----------------------------------------------------------------------------------------
%	RESUMEN PROFESIONAL
%----------------------------------------------------------------------------------------
\section{Resumen Profesional}

\cvitem{}{Líder con más de \textbf{30 años de experiencia} en administración pública, gestión de programas gubernamentales, políticas agropecuarias y desarrollo rural sostenible. Experto en \textbf{administración de recursos humanos, materiales y financieros} a gran escala, habiendo dirigido equipos de hasta 320 personas y ejercido presupuestos superiores a 250 MDP anuales. Experiencia en \textbf{sistemas de gestión de calidad ISO 9000}: desarrollo de POE y procesos estandarizados para registro de productos y logística de exportación. Reciente colaborador del equipo SADER Yucatán en el desarrollo de plataformas digitales para seguimiento zoosanitario. Comprometido con el bienestar de las familias yucatecas mediante la planeación estratégica y la innovación institucional.}

%----------------------------------------------------------------------------------------
%	COMPETENCIAS ADMINISTRATIVAS
%----------------------------------------------------------------------------------------
\section{Competencias Administrativas Clave}

\cvitem{Gestión de Personal}{Liderazgo de equipos multidisciplinarios de hasta \textbf{320+ servidores públicos} en 22 oficinas regionales}
\cvitem{Administración Presupuestal}{Ejercicio y supervisión de \textbf{250+ MDP anuales} en programas federales-estatales con cumplimiento normativo}
\cvitem{Operación de Programas}{Distribución de \textbf{120 MDP anuales} del programa Procampo a 45,000+ beneficiarios}
\cvitem{Coordinación Interinstitucional}{Vinculación con SADER, SENASICA, APHIS-USDA, gobiernos estatales y organismos auxiliares}
\cvitem{Gestión Zoosanitaria}{Trazabilidad SINIIGA-SINIDA, campañas de tuberculosis bovina, certificación para exportación}
\cvitem{Innovación Digital}{Desarrollo web GOB.mx v3, Firebase, VS Code, sistemas de seguimiento de acuerdos}

%----------------------------------------------------------------------------------------
%	EXPERIENCIA PROFESIONAL
%----------------------------------------------------------------------------------------
\section{Experiencia Profesional}

\cventry{05/2025--Presente}{Consultor Independiente (Asesor Externo)}{SADER Yucatán}{Mérida}{}{
Colaboración en el equipo del MVZ Jorge Carlos Berlín Montero, Representante Estatal.
\begin{itemize}
\item Desarrollé el \textbf{``Centro de Consulta de Acuerdos Zoosanitarios''} (ceso-aphis-yuc.web.app), plataforma web bilingüe bajo estándares GOB.mx v3 para seguimiento del CESO y APHIS-USDA/SENASICA
\item Implementé migración de 9 páginas HTML, estandarización de 50+ elementos UI, paneles de control en tiempo real y sistema de carga de evidencias
\item Facilité el cumplimiento normativo y la cooperación bilateral México-EE.UU. en exportaciones pecuarias
\end{itemize}}

\cventry{08/2020--04/2025}{Gerente de Proyectos}{Youcatan Land SAPI de CV}{Mérida}{}{
\begin{itemize}
\item Lideré la obtención de concesión estatal para ``Tabi, Santuario de Selva Maya'' en ANP San Juan Bautista Tabi (1,350 ha)
\item Desarrollé el ``Programa de Estrategias Integradas para la Sustentabilidad Agrícola'' para Blue Core A.C., promoviendo transición agroecológica en comunidades rurales
\end{itemize}}

\cventry{09/2017--12/2025}{Profesor, Lic. en Administración de Recursos Naturales}{Universidad Marista de Mérida}{Mérida}{}{
\begin{itemize}
\item Impartí ``Análisis y Evaluación de Sistemas de Producción Animal'' y ``Diseño de Sistemas Ganaderos Sostenibles'' a alumnos de 9° y 10° semestre
\item Formé especialistas en prácticas agropecuarias sostenibles y conservación
\end{itemize}}

\cventry{10/2013--08/2017}{Economista Agrícola, Estrategia M-REDD+}{The Nature Conservancy}{Mérida}{Alianza México REDD+}{
\begin{itemize}
\item Implementé sistemas ganaderos silvopastoriles en 8 ranchos piloto para desarrollo rural bajo en carbono
\item Secretario Ejecutivo de la Comunidad de Práctica de Ganadería Sostenible
\end{itemize}}

\cventry{08/2012--02/2013}{Encargado del Despacho de la Delegación}{SAGARPA}{Yucatán}{}{
\textbf{Máxima autoridad federal agropecuaria en el Estado de Yucatán.}
\begin{itemize}
\item \textbf{Lideré un equipo de 320+ servidores públicos} distribuidos en 22 oficinas regionales
\item Coordiné la operación de todos los programas federales agropecuarios y las relaciones institucionales con el Gobierno del Estado
\item Representé a la institución ante autoridades estatales, productores y organismos del sector
\end{itemize}}

\cventry{10/2005--09/2013}{Subdelegado de Planeación y Desarrollo Rural}{SAGARPA}{Yucatán}{}{
\begin{itemize}
\item \textbf{Administré el programa Procampo}: distribución de 120 MDP anuales a más de 45,000 productores
\item \textbf{Supervisé el ejercicio de 250 MDP anuales} en programas federales-estatales concurrentes
\item Aseguré cumplimiento legal, disciplina fiscal y transparencia en el uso de recursos públicos
\item Coordiné la planeación estratégica del desarrollo rural en el Estado
\end{itemize}}

\cventry{08/2002--09/2005}{Jefe de Distrito de Desarrollo Rural, Ticul}{SAGARPA}{Yucatán}{}{
\begin{itemize}
\item Coordiné programas de subsidios para más de \textbf{18,000 productores anuales}
\item \textbf{Administré un equipo de 148 servidores públicos}
\item Establecí \textbf{18 Consejos Municipales de Desarrollo Rural Sustentable}
\end{itemize}}

\cventry{08/1998--07/2001}{Propietario}{AGROPHARMA}{Yucatán}{Microempresa Farmacéutica Veterinaria}{
\begin{itemize}
\item Incrementé ventas de 150,000 a \textbf{2 millones MXN anuales}
\item Lancé 15 nuevos productos al mercado regional
\end{itemize}}

\cventry{01/1994--12/1997}{Gerente de Exportación}{LAPISA, SA de CV}{Michoacán}{Laboratorio Farmacéutico Veterinario}{
\begin{itemize}
\item \textbf{Diseñé y establecí el departamento de exportación} desde cero
\item Inicié operaciones comerciales en \textbf{13 países} de América Latina y Asia
\item \textbf{Participé en el proceso de certificación ISO 9000}: desarrollé POE (Procedimientos Operativos Estándar) para registro de productos farmacéuticos veterinarios en el extranjero y logística de exportación
\end{itemize}}

%----------------------------------------------------------------------------------------
%	EDUCACIÓN
%----------------------------------------------------------------------------------------
\section{Educación}

\cventry{2004--2007}{Maestría en Gobierno y Políticas Públicas (estudios concluidos)}{Universidad Autónoma de Yucatán}{Mérida}{}{}
\cventry{1994--1997}{Maestría en Administración de Empresas (estudios concluidos)}{Universidad del Valle de Atemajac}{La Piedad, Michoacán}{}{}
\cventry{1984--1988}{Licenciatura en Medicina Veterinaria y Zootecnia}{Universidad Nacional Autónoma de México}{Ciudad de México}{Título y Cédula Profesional}{}

%----------------------------------------------------------------------------------------
%	IDIOMAS
%----------------------------------------------------------------------------------------
\section{Idiomas}

\cvitemwithcomment{Español}{Nativo}{}
\cvitemwithcomment{Inglés}{Competente bilingüe}{Comunicación profesional oral y escrita}
\cvitemwithcomment{Francés}{Intermedio}{}

%----------------------------------------------------------------------------------------
%	HABILIDADES TÉCNICAS
%----------------------------------------------------------------------------------------
\section{Habilidades Técnicas}

\cvitem{Desarrollo Web}{HTML, CSS, JavaScript, Firebase, estándares GOB.mx v3}
\cvitem{Herramientas}{Visual Studio Code, QGIS, Microsoft Office, Google Workspace}
\cvitem{IA y Análisis}{Herramientas de Inteligencia Artificial, análisis de datos, dashboards}

\end{document}
