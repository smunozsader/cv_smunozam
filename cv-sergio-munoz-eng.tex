%% Professional CV - Sergio Muñoz de Alba Medrano, DVM
%% English Version - December 2025
%% Compile with: pdflatex cv-sergio-munoz-eng.tex

\documentclass[10pt,letterpaper,sans]{moderncv}

% moderncv style
\moderncvstyle{banking}
\moderncvcolor{blue}

% Encoding and fonts
\usepackage[utf8]{inputenc}
\usepackage[T1]{fontenc}
\usepackage{lmodern}

% Margin adjustment
\usepackage[scale=0.85,footskip=40pt]{geometry}
\setlength{\hintscolumnwidth}{2.5cm}

% Reduce name and title font size
\renewcommand*{\namefont}{\fontsize{20}{22}\mdseries\upshape}
\renewcommand*{\titlefont}{\fontsize{14}{16}\mdseries\upshape}

% Personal data
\name{Sergio}{Muñoz de Alba Medrano}
\title{~\\Doctor of Veterinary Medicine}
\address{Mérida, Yucatán}{Mexico}{}
\email{smunozam@gmail.com}
\phone[mobile]{+52~999~200~5550}
\homepage{tinyurl.com/linkedinSMAM}
\extrainfo{Specialist in Public Administration, Agricultural Policy and Rural Development}

%----------------------------------------------------------------------------------------
\begin{document}
\makecvtitle

%----------------------------------------------------------------------------------------
%	PROFESSIONAL SUMMARY
%----------------------------------------------------------------------------------------
\section{Professional Summary}

\cvitem{}{Leader with over \textbf{30 years of experience} in public administration, government program management, agricultural policy, and sustainable rural development. Expert in \textbf{human resources, material, and financial administration} at scale, having led teams of up to 320 people and managed budgets exceeding 250 million MXN annually. Experienced in \textbf{ISO 9000 quality management systems}: SOP development and standardized processes for product registration and export logistics. Recent collaborator with SADER Yucatán team in developing digital platforms for animal health monitoring. Committed to the welfare of Yucatecan families through strategic planning and institutional innovation.}

%----------------------------------------------------------------------------------------
%	KEY ADMINISTRATIVE COMPETENCIES
%----------------------------------------------------------------------------------------
\section{Key Administrative Competencies}

\cvitem{Personnel Management}{Leadership of multidisciplinary teams of up to \textbf{320+ public servants} across 22 regional offices}
\cvitem{Budget Administration}{Oversight and execution of \textbf{250+ million MXN annually} in federal-state programs with regulatory compliance}
\cvitem{Program Operations}{Distribution of \textbf{120 million MXN annually} through Procampo program to 45,000+ beneficiaries}
\cvitem{Interagency Coordination}{Liaison with SADER, SENASICA, APHIS-USDA, state governments, and auxiliary organizations}
\cvitem{Animal Health Management}{SINIIGA-SINIDA traceability, bovine tuberculosis campaigns, export certification}
\cvitem{Digital Innovation}{GOB.mx v3 web development, Firebase, VS Code, agreement tracking systems}

%----------------------------------------------------------------------------------------
%	PROFESSIONAL EXPERIENCE
%----------------------------------------------------------------------------------------
\section{Professional Experience}

\cventry{05/2025--Present}{Independent Consultant (External Advisor)}{SADER Yucatán}{Mérida}{}{
Collaboration with MVZ Jorge Carlos Berlín Montero's team, State Representative.
\begin{itemize}
\item Developed the \textbf{``Centro de Consulta de Acuerdos Zoosanitarios''} (ceso-aphis-yuc.web.app), a bilingual web platform under GOB.mx v3 standards for CESO and APHIS-USDA/SENASICA tracking
\item Implemented migration of 9 HTML pages, standardization of 50+ UI elements, real-time dashboards, and evidence upload system
\item Facilitated regulatory compliance and bilateral Mexico-U.S. cooperation in livestock exports
\end{itemize}}

\cventry{08/2020--04/2025}{Project Manager}{Youcatan Land SAPI de CV}{Mérida}{}{
\begin{itemize}
\item Led efforts to secure state concession for ``Tabi, Santuario de Selva Maya'' in San Juan Bautista Tabi Protected Area (1,350 ha)
\item Developed the ``Integrated Strategies Program for Agricultural Sustainability'' for Blue Core A.C., promoting agroecological transition in rural communities
\end{itemize}}

\cventry{09/2017--12/2025}{Professor, Natural Resources Management Programme}{Universidad Marista de Mérida}{Mérida}{}{
\begin{itemize}
\item Taught ``Analysis and Evaluation of Animal Production Systems'' and ``Design of Sustainable Livestock Systems'' to 9th and 10th semester students
\item Trained specialists in sustainable agricultural practices and conservation
\end{itemize}}

\cventry{10/2013--08/2017}{Agricultural Economist, M-REDD+ Strategy}{The Nature Conservancy}{Mérida}{Alianza México REDD+}{
\begin{itemize}
\item Implemented silvopastoral livestock systems in 8 pilot ranches for low-carbon rural development
\item Executive Secretary of the Sustainable Cattle Production Community of Practice
\end{itemize}}

\cventry{08/2012--02/2013}{Acting Head of Delegation}{SAGARPA}{Yucatán}{}{
\textbf{Highest federal agricultural authority in the State of Yucatán.}
\begin{itemize}
\item \textbf{Led a team of 320+ public servants} across 22 regional offices
\item Coordinated all federal agricultural program operations and institutional relations with the State Government
\item Represented the institution before state authorities, producers, and sector organizations
\end{itemize}}

\cventry{10/2005--09/2013}{Subdelegate for Planning and Rural Development}{SAGARPA}{Yucatán}{}{
\begin{itemize}
\item \textbf{Managed the Procampo program}: distribution of 120 million MXN annually to over 45,000 producers
\item \textbf{Oversaw execution of 250 million MXN annually} in concurrent federal-state programs
\item Ensured legal compliance, fiscal discipline, and transparency in public resource management
\item Coordinated strategic planning for rural development in the State
\end{itemize}}

\cventry{08/2002--09/2005}{Head of Rural Development District, Ticul}{SAGARPA}{Yucatán}{}{
\begin{itemize}
\item Coordinated subsidy programs for over \textbf{18,000 producers annually}
\item \textbf{Managed a team of 148 public servants}
\item Established \textbf{18 Municipal Councils for Sustainable Rural Development}
\end{itemize}}

\cventry{08/1998--07/2001}{Owner}{AGROPHARMA}{Yucatán}{Veterinary Pharmaceutical Microenterprise}{
\begin{itemize}
\item Increased sales from 150,000 to \textbf{2 million MXN annually}
\item Launched 15 new products to the regional market
\end{itemize}}

\cventry{01/1994--12/1997}{Export Manager}{LAPISA, SA de CV}{Michoacán}{Veterinary Pharmaceutical Laboratory}{
\begin{itemize}
\item \textbf{Designed and established the export department} from scratch
\item Initiated commercial operations in \textbf{13 countries} across Latin America and Asia
\item \textbf{Participated in ISO 9000 certification process}: developed SOPs (Standard Operating Procedures) for international registration of veterinary pharmaceutical products and export logistics
\end{itemize}}

%----------------------------------------------------------------------------------------
%	EDUCATION
%----------------------------------------------------------------------------------------
\section{Education}

\cventry{2004--2007}{Master's in Government and Public Policy (coursework completed)}{Universidad Autónoma de Yucatán}{Mérida}{}{}
\cventry{1994--1997}{Master's in Business Administration (coursework completed)}{Universidad del Valle de Atemajac}{La Piedad, Michoacán}{}{}
\cventry{1984--1988}{Bachelor's in Veterinary Medicine and Animal Science}{Universidad Nacional Autónoma de México}{Mexico City}{Professional License}{}

%----------------------------------------------------------------------------------------
%	LANGUAGES
%----------------------------------------------------------------------------------------
\section{Languages}

\cvitemwithcomment{Spanish}{Native}{}
\cvitemwithcomment{English}{Bilingual Proficiency}{Professional oral and written communication}
\cvitemwithcomment{French}{Intermediate}{}

%----------------------------------------------------------------------------------------
%	TECHNICAL SKILLS
%----------------------------------------------------------------------------------------
\section{Technical Skills}

\cvitem{Web Development}{HTML, CSS, JavaScript, Firebase, GOB.mx v3 standards}
\cvitem{Tools}{Visual Studio Code, QGIS, Microsoft Office, Google Workspace}
\cvitem{AI \& Analytics}{Artificial Intelligence tools, data analysis, dashboards}

\end{document}
