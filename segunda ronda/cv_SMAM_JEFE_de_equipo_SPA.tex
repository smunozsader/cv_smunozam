%% Curriculum Vitae Profesional - Sergio Muñoz de Alba Medrano, MVZ
%% Versión en Español – 3 páginas, sin eliminar puestos ni formación
%% Compilar con: pdflatex cv-sergio-3page-spa.tex

\documentclass[9pt,letterpaper,sans]{moderncv}

\moderncvstyle{banking}
\moderncvcolor{blue}

\usepackage[utf8]{inputenc}
\usepackage[T1]{fontenc}
\usepackage{lmodern}

% Layout máximo comprimido
\usepackage[scale=0.75,footskip=24pt,top=0.6in,bottom=0.5in]{geometry}
\setlength{\hintscolumnwidth}{2cm}

\renewcommand*{\namefont}{\fontsize{16}{18}\mdseries\upshape}
\renewcommand*{\titlefont}{\fontsize{11}{12}\mdseries\upshape}

\setlength{\parskip}{1pt}
\setlength{\itemsep}{0pt}
\setlength{\parsep}{0pt}
\setlength{\topskip}{2pt}
\setlength{\parindent}{0pt}

% Datos personales
\name{Sergio}{Muñoz de Alba Medrano}
\title{Médico Veterinario Zootecnista}
\address{Mérida, Yucatán}{México}{}
\email{smunozam@gmail.com}
\phone[mobile]{+52~999~200~5550}
\homepage{tinyurl.com/linkedinSMAM}
\extrainfo{Dirección de Operaciones y Programas | Administración Pública | Desarrollo Organizacional}

\begin{document}
\makecvtitle

% RESUMEN PROFESIONAL
\section{Resumen Profesional}

\cvitem{}{Directivo en operaciones y gestión de programas con \textbf{más de 30 años de experiencia} coordinando programas públicos de gran escala, iniciativas complejas y equipos multidisciplinarios en gobierno, organizaciones no lucrativas y proyectos privados orientados al desarrollo regional y la sostenibilidad. Ha administrado \textbf{presupuestos superiores a 250 millones de MXN anuales}, liderado equipos de \textbf{más de 300 personas} y coordinado \textbf{más de 20 oficinas regionales}, implementando estructuras, procesos y sistemas que mejoran la ejecución, la transparencia y el cumplimiento normativo. Se desempeña como \textbf{socio tipo Chief of Staff} para la alta dirección, alineando la estrategia con la operación diaria, ordenando flujos de información y asegurando que planes, personas y recursos avancen en la misma dirección. Cuenta con una sólida trayectoria en administración pública y desarrollo sostenible y está listo para aplicar estas capacidades de gestión en sectores diversos, más allá del ámbito de la conservación.}

% COMPETENCIAS CLAVE DE GESTIÓN
\section{Competencias Clave de Gestión}

\cvitem{Liderazgo de Operaciones}{Dirección de equipos multidisciplinarios de hasta \textbf{320+ personas} en 22 oficinas regionales, con objetivos claros, seguimiento y evaluación de resultados.}
\cvitem{Gestión de Programas}{Ejecución de portafolios multi–programa con presupuestos anuales superiores a \textbf{250+ millones de MXN} y más de \textbf{45,000 beneficiarios}, alineados con las metas estratégicas.}
\cvitem{Gestión Financiera}{Supervisión de presupuestos, asignación de recursos, documentación de soporte y acompañamiento en auditorías y procesos de transparencia en entornos públicos y no lucrativos.}
\cvitem{Desarrollo Organizacional}{Creación de consejos, comunidades de práctica y equipos interfuncionales para fortalecer la colaboración, la gobernanza y la gestión del conocimiento.}
\cvitem{Procesos y Calidad}{Diseño e implementación de procedimientos estandarizados (SOPs), participación en sistemas de calidad ISO 9000 y mejora continua de flujos de trabajo y controles.}
\cvitem{Relación con Actores Clave}{Enlace con gobiernos federal y estatal, agencias internacionales, productores, organizaciones de la sociedad civil y socios del sector privado.}
\cvitem{Sistemas Digitales y Datos}{Uso de plataformas web, sistemas de trazabilidad y seguimiento de acuerdos, y tableros (normativa GOB.mx v3) para gestionar información y apoyar la toma de decisiones.}

% EXPERIENCIA PROFESIONAL
\section{Experiencia Profesional}

\cventry{05/2025--Presente}{Consultor Independiente (Asesor Externo)}{SADER Yucatán}{Mérida}{}{
Colaboración con el equipo del Representante Estatal para fortalecer sistemas operativos, gestión de datos y esquemas de gobernanza en sanidad animal y acuerdos bilaterales.
\begin{itemize}
\item Lideró el diseño e implementación del \textbf{“Centro de Consulta de Acuerdos Zoosanitarios”} (ceso-aphis-yuc.web.app), sistema bilingüe de seguimiento de acuerdos y compromisos entre CESO y APHIS-USDA/SENASICA.
\item Estructuró modelos de datos, estandarizó campos de información y definió flujos de trabajo para asegurar el registro, seguimiento y consulta precisa de acuerdos bilaterales y sus evidencias.
\item Coordinó actores federales, estatales e internacionales para alinear procedimientos, requisitos documentales y calendarios de actualización, mejorando trazabilidad y preparación ante auditorías.
\item Desarrolló tableros y vistas de reporte para monitorear estatus de acuerdos, acciones pendientes e historial de cumplimiento, apoyando la toma de decisiones y la transparencia institucional.
\end{itemize}}

\cventry{08/2020--04/2025}{Gerente de Proyectos}{Youcatan Land SAPI de CV}{Mérida}{}{
Gestión de proyectos estratégicos vinculados con conservación, manejo de territorio y desarrollo rural.
\begin{itemize}
\item Lideró el proceso administrativo y técnico para obtener la concesión estatal del \textbf{"Tabi, Santuario de Selva Maya"} (1,350 ha), coordinando documentación, requisitos y seguimiento ante dependencias gubernamentales.
\item Diseñó el \textbf{"Programa de Estrategias Integrales para la Sustentabilidad Agrícola"} para Blue Core A.C., definiendo objetivos, indicadores y mecanismos de monitoreo para la transición agroecológica en comunidades rurales.
\end{itemize}}

\cventry{09/2017--12/2025}{Profesor, Programa de Gestión de Recursos Naturales}{Universidad Marista de Mérida}{Mérida}{}{
Profesor de tiempo parcial en el programa de Gestión de Recursos Naturales.
\begin{itemize}
\item Impartió “Análisis y Evaluación de Sistemas de Producción Animal” y “Diseño de Sistemas Ganaderos Sustentables” a estudiantes de últimos semestres.
\item Diseñó programas de curso, criterios de evaluación y actividades de aprendizaje basadas en proyectos, integrando productividad, sustentabilidad y cumplimiento regulatorio.
\item Asesoró a estudiantes en proyectos aplicados, vinculando conceptos técnicos con retos operativos y organizacionales del mundo real.
\end{itemize}}

\cventry{10/2013--08/2017}{Economista Agrícola, Estrategia M-REDD+}{The Nature Conservancy}{Mérida}{Alianza México REDD+}{
Apoyo a la implementación de estrategias de desarrollo rural de bajas emisiones.
\begin{itemize}
\item Coordinó la implementación de sistemas silvopastoriles en 8 ranchos piloto, organizando asistencia técnica, planes de trabajo y monitoreo con productores.
\item Se desempeñó como Secretario Ejecutivo de la Comunidad de Práctica de Ganadería Sustentable, gestionando flujos de información, reuniones y documentación entre organizaciones.
\item Contribuyó a informes y gestión del conocimiento, asegurando que información de campo y lecciones aprendidas se capturaran y organizaran para uso del programa.
\end{itemize}}

\cventry{08/2012--02/2013}{Encargado del Despacho de la Delegación}{SAGARPA}{Yucatán}{}{
\textbf{Máxima autoridad agrícola federal en el estado de Yucatán}, fungiendo como Director Regional y responsable principal de la operación institucional.
\begin{itemize}
\item Brindó liderazgo estratégico y operativo a \textbf{más de 320 servidores públicos} en \textbf{22 oficinas regionales}, asegurando continuidad de servicios y entrega de programas.
\item Coordinó la operación de todos los programas agrícolas federales en Yucatán, alineando iniciativas, equipos y recursos bajo una sola dirección operativa.
\item Condujo las relaciones institucionales con el Gobierno del Estado, organizaciones de productores y actores del sector, como principal enlace y representante de la institución federal.
\item Supervisó la implementación de políticas nacionales a nivel estatal, monitoreando desempeño y asegurando apego a lineamientos y normas federales.
\end{itemize}}

\cventry{10/2005--09/2013}{Subdelegado de Planeación y Desarrollo Rural}{SAGARPA}{Yucatán}{}{
Responsable de la planeación y operación de los programas agrícolas federales en Yucatán.
\begin{itemize}
\item Gestionó el \textbf{programa Procampo}, supervisando el flujo anual de \textbf{120 millones de MXN} a más de \textbf{45,000 productores}, incluyendo administración de padrones, validación de elegibilidad y ejecución de pagos.
\item Supervisó la ejecución de más de \textbf{250 millones de MXN anuales} en programas concurrentes federación–estado bajo la \textbf{“Alianza para el Campo”}, alineando presupuestos, cartera de proyectos y reportes con reglas de operación y anexos técnicos.
\item Implementó procedimientos y mecanismos de control para asegurar cumplimiento normativo, disciplina fiscal y transparencia en el uso de recursos públicos.
\item Lideró procesos de planeación estratégica para el desarrollo rural, integrando información de múltiples programas para apoyar decisiones, monitoreo y evaluación.
\item Coordinó con autoridades estatales y organizaciones de productores para resolver cuellos de botella y supervisó la integración mensual de información estadística de CADER y Distritos de Desarrollo Rural en coordinación con SIAP.
\end{itemize}}

\cventry{08/2002--09/2005}{Jefe de Distrito de Desarrollo Rural, Ticul}{SAGARPA}{Yucatán}{}{
Responsable operativo regional de los programas federales de desarrollo rural en el distrito de Ticul.
\begin{itemize}
\item Coordinó programas de apoyo para más de \textbf{18,000 productores al año}, asegurando registro adecuado, verificación de elegibilidad y seguimiento a la entrega de apoyos.
\item \textbf{Administró un equipo de 148 servidores públicos}, organizando flujos de trabajo, distribuyendo tareas y monitoreando desempeño en varios municipios.
\item Estableció \textbf{18 Consejos Municipales para el Desarrollo Rural Sustentable}, creando estructuras formales de participación y supervisión de programas.
\end{itemize}}

\cventry{08/1998--07/2001}{Propietario}{AGROPHARMA}{Yucatán}{Microempresa Farmacéutica Veterinaria}{
Emprendedor y director general de una microempresa farmacéutica veterinaria.
\begin{itemize}
\item Incrementó ventas anuales de \textbf{150,000 a 2 millones de MXN}, ampliando base de clientes y penetración de productos.
\item Lanzó \textbf{15 nuevos productos}, coordinando formulación, registro, empaque, precios y estrategia de salida al mercado.
\item Gestionó relaciones con proveedores, inventarios y controles básicos de calidad para asegurar disponibilidad y confiabilidad de productos.
\item Supervisó procesos administrativos, financieros y comerciales, incluyendo facturación, cobranza y atención a clientes.
\end{itemize}}

\cventry{01/1994--12/1997}{Gerente de Desarrollo de Nuevos Productos y Exportación}{LAPISA, S.A. de C.V.}{Michoacán}{Laboratorio Farmacéutico Veterinario}{
Responsable conjunto de desarrollo de nuevos productos y operaciones de exportación internacional.
\begin{itemize}
\item \textbf{Diseñó y estableció el departamento de exportación} desde cero, incluyendo procedimientos, formatos y mecanismos de coordinación con áreas internas y clientes internacionales.
\item Inició operaciones comerciales en \textbf{13 países} de América Latina y Asia, gestionando logística, embarques y requisitos regulatorios para la entrada de productos.
\item Lideró la elaboración de expedientes regulatorios y fungió como enlace con ministerios de agricultura y agencias regulatorias para obtener registros zoosanitarios.
\item \textbf{Participó en procesos de certificación ISO 9000}, desarrollando SOPs para registro internacional de productos y logística de exportación.
\item Coordinó equipos interfuncionales (I+D, producción, calidad, ventas) para asegurar cumplimiento de criterios técnicos y regulatorios.
\end{itemize}}

% FORMACIÓN ACADÉMICA
\section{Formación Académica}

\cventry{2004--2007}{Maestría en Gobierno y Políticas Públicas (cursos concluidos)}{Universidad Autónoma de Yucatán}{Mérida}{}{}
\cventry{1994--1997}{Maestría en Administración de Empresas (cursos concluidos)}{Universidad del Valle de Atemajac}{La Piedad, Michoacán}{}{}
\cventry{1984--1988}{Licenciatura en Medicina Veterinaria y Zootecnia}{Universidad Nacional Autónoma de México}{Ciudad de México}{Cédula profesional}{}

% IDIOMAS
\section{Idiomas}

\cvitemwithcomment{Español}{Nativo}{}
\cvitemwithcomment{Inglés}{Bilingüe}{Comunicación oral y escrita a nivel profesional}
\cvitemwithcomment{Francés}{Intermedio}{}

% HABILIDADES TÉCNICAS
\section{Habilidades Técnicas}

\cvitem{Desarrollo Web y Datos}{HTML, CSS, JavaScript, Vue.js, Firebase, estándares GOB.mx v3; tableros y aplicaciones web (herramientas de calidad, sistemas financieros, seguimiento de datos).}
\cvitem{Herramientas de Desarrollo}{Visual Studio Code, Git/GitHub, Node.js; análisis de datos, QGIS, Microsoft Office, Google Workspace.}
\cvitem{IA y Automatización}{GitHub Copilot y asistentes de IA para redacción de documentos, estructuración de información y aceleración de análisis.}

\end{document}
